\documentclass[a4paper,11pt]{article}
% !!!! ----- A compiler avec XeLaTeX ou LuaLaTeX ----- !!!!
\usepackage[top=21mm,bottom=28mm,left=28mm,right=21mm,nohead,nofoot]{geometry}
\usepackage{tikz}
\usepackage{fontspec}
\defaultfontfeatures{Ligatures=TeX}
\usepackage{labopp}
\usepackage{fancyvrb}
\pagestyle{empty}
\begin{document}

% ------------------- l'en-tête ------------------------
\laboheader

\def\CommentChar{\char37}
\catcode`!=\active
\begin{Verbatim}[defineactive=\def!{\color{orange}\CommentChar}]
\documentclass[a4paper,11pt]{article}
\usepackage[top=21mm,bottom=28mm,left=28mm,right=21mm,nohead,nofoot]{geometry}
\usepackage{tikz}
\usepackage{labopp} ! <----------- le package avec les commandes du labo
\begin{document}
\laboheader ! <----------- l'en-tête en haut a gauche
\end{Verbatim}

% ------------------- test ------------------------

\begin{Verbatim}[defineactive=\def!{\color{orange}\CommentChar}]
! ---------------------------------- exemple 1 (valeurs par défaut)
\logonompp{2}
\end{Verbatim}
\logonompp{2}

\begin{Verbatim}[defineactive=\def!{\color{orange}\CommentChar}]
! ---------------------------------- exemple 2 (logo seul + choix des tailles)
\logopp{1}\logopp{2}\logopp{4}
\end{Verbatim}
\logopp{1}\logopp{2}\logopp{4}


\begin{Verbatim}[defineactive=\def!{\color{orange}\CommentChar}]
! ---------------------------------- exemple 3 (la totale)
\def\setup{[fill=black]}
\def\setdown{[shade, top color=bleuclaireuniv, bottom color=bleufonceuniv]}
\def\setborder{[draw=black]}
\def\setcircles{[draw=gray, thin]}
\def\settextlabo{[shade, top color=bleuclaireuniv, bottom color=bleufonceuniv]}
\logonompp{2}
\end{Verbatim}
\def\setup{[fill=black]}
\def\setdown{[shade, top color=bleuclaireuniv, bottom color=bleufonceuniv]}
\def\setborder{[draw=black]}
\def\setcircles{[draw=gray, thin]}
\def\settextlabo{[shade, top color=bleuclaireuniv, bottom color=bleufonceuniv]}
\logonompp{2}



\vfill

\begin{Verbatim}[defineactive=\def!{\color{orange}\CommentChar}]
! ---------------------------------- exemple 4 (le bas de page)
\labofooter
\end{Verbatim}
\labofooter

\pagebreak

% ----------------- en tête avec le coordonnées du labo ----------------
\setdefaultspp
\labotextheader
\vspace{4cm}

\begin{Verbatim}[defineactive=\def!{\color{orange}\CommentChar}]
! ----------------- (l'en-tête avec les coordonnées du labo)
\labotextheader

! ----------------- Les couleurs définies dans la package :
\end{Verbatim}

% ----------------- les couleurs ----------------
\def\drawcolor#1{
\tikz{\path[fill=#1] (0,0) node[above, rectangle,minimum width=3cm,rounded corners=1mm, draw=#1]{#1} (-1.5,-1) rectangle (1.5,0);}}



\drawcolor{bleuclaireuniv}
\drawcolor{bleufonceuniv}
\drawcolor{noiruniv}
\drawcolor{grisuniv}
\vspace{4mm}

\drawcolor{vertinsmi}
\drawcolor{vertfonce}
\drawcolor{vertclaire}

\vfill
\begin{Verbatim}[defineactive=\def!{\color{orange}\CommentChar}]
! ----------------- (le bas de page dans le syle de l'Université)
\labounivfooter
\end{Verbatim}
\labounivfooter


\end{document}
