\documentclass[a4paper,11pt]{article}
\usepackage[top=21mm,bottom=28mm,left=28mm,right=21mm,nohead,nofoot]{geometry}
\usepackage{tikz}
\usepackage{labopp}
\usepackage{fancyvrb}
\pagestyle{empty}
\begin{document}

% ------------------- l'en-t\^ete ------------------------
\laboheader

\def\CommentChar{\char37}
\catcode`!=\active
\begin{Verbatim}[defineactive=\def!{\color{orange}\CommentChar}]
\documentclass[a4paper,11pt]{article}
\usepackage[top=21mm,bottom=28mm,left=28mm,right=21mm,nohead,nofoot]{geometry}
\usepackage{tikz} ! <----------- pas obligatoire car labopp le charge si necessaire
\usepackage{labopp} ! <----------- le package avec les commandes du labo
\begin{document}
\laboheader ! <----------- l'en-tete en haut a gauche
\end{Verbatim}

% ------------------- test ------------------------

\begin{Verbatim}[defineactive=\def!{\color{orange}\CommentChar}]
! ---------------------------------- exemple 1 (valeurs par defaut)
\logonompp{2}
\end{Verbatim}
\logonompp{2}

\begin{Verbatim}[defineactive=\def!{\color{orange}\CommentChar}]
! ---------------------------------- exemple 2 (logo seul + choix des tailles)
\logopp{1}\logopp{2}\logopp{4}
\end{Verbatim}
\logopp{1}\logopp{2}\logopp{4}


\begin{Verbatim}[defineactive=\def!{\color{orange}\CommentChar}]
! ---------------------------------- exemple 3 (la totale)
\tikzset{
  up/.style={fill=black},
  down/.style={shade, top color=bleuclaireuniv, bottom color=bleufonceuniv},
  border/.style={draw=black},
  circles/.style={draw=gray},
  textlabo/.style={shade, top color=bleuclaireuniv, bottom color=bleufonceuniv}
}
\logonompp{2}
\end{Verbatim}
{ % dans un groupe les changement de style sont locaux
  \tikzset{
    up/.style={fill=black},
    down/.style={shade, top color=bleuclaireuniv, bottom color=bleufonceuniv},
    border/.style={draw=black},
    circles/.style={draw=gray},
    textlabo/.style={shade, top color=bleuclaireuniv, bottom color=bleufonceuniv}
  }
  \logonompp{2}
}


\vfill

\begin{Verbatim}[defineactive=\def!{\color{orange}\CommentChar}]
! ---------------------------------- exemple 4 (le bas de page)
\labofooter
\end{Verbatim}
\labofooter

\pagebreak

% ----------------- en-t\^ete avec le coordonn\'ees du labo ----------------
\labotextheader
\vspace{4cm}

\begin{Verbatim}[defineactive=\def!{\color{orange}\CommentChar}]
! ----------------- (l'en-tete avec les coordonnees du labo)
\labotextheader

! ----------------- Les couleurs d\'efinies dans la package :
\end{Verbatim}

% ----------------- les couleurs ----------------
\def\drawcolor#1{
\tikz{\path[fill=#1] (0,0) node[above, rectangle,minimum width=3cm,rounded corners=1mm, draw=#1]{#1} (-1.5,-1) rectangle (1.5,0);}}

\drawcolor{vertfonce}
\drawcolor{vertclaire}
\drawcolor{vertinsmi}
\drawcolor{hibiscusuniv}
\vspace{4mm}

\drawcolor{bleuclaireuniv}
\drawcolor{bleufonceuniv}
\drawcolor{noiruniv}
\drawcolor{grisuniv}

\vfill
\begin{Verbatim}[defineactive=\def!{\color{orange}\CommentChar}]
! ----------------- (le bas de page dans le syle de l'universite)
\labounivfooter
\end{Verbatim}
\labounivfooter


\end{document}
