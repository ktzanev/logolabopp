%%%%%%%%%%%%%%%%%%%%%%%%%%%%%%%%%%%%%%%%%%%%%%%%%%%%%%%%%%%%%%%%%%%%%%%%
%
%  pour personnaliser (exemples) :
%
%   \setlength{\destlen}{7cm} % largeur pour l'adresse du d\'estinataire
%   \setlength{\explen}{7cm} % largeur pour la signature de l'expeditaur
%
%  pour mettre l'en-t\^ete ou/et le pied de page \`a chaque page :
%   \usepackage{everypage}
%   et apr\`es par exemple :
%    \AddEverypageHook{\laboheader}
%    \AddThispageHook{\textheader{...}}
%
%  ou pour mettre que sur les pages impaires
%   \usepackage{everypage}
%   \usepackage{ifthen}
%   et apr\`es
%    \AddEverypageHook{\ifthenelse{\isodd{\thepage}}{\laboheader \labofooter}{}}
%    \AddThispageHook{\textheader{...}}
%
\documentclass[a4paper,11pt]{article}
\usepackage[top=42mm,left=28mm,right=21mm,nohead,nofoot]{geometry}
% --------------- PDFLaTeX, XeLaTeX ou LuaLaTeX ?
\usepackage{iftex}
\ifPDFTeX
  % --------------- PDFLaTeX ou LaTeX
  \usepackage[utf8]{inputenc} % pour avoir ce fichier en utf8
  \usepackage[T1]{fontenc} % pour avoir les lettres accentuées dans le résultat PDF
  \usepackage{lmodern} % pour utiliser cette police vectorielle (elle utilise l'encodage T1 avec 256 lettres)
\else
  % --------------- XeLaTeX ou LuaLaTeX
  \usepackage{fontspec} % pour le bon encodage des PDFs => police lmodern
\fi
\usepackage[french]{babel}
\usepackage{csquotes}
\usepackage{laboppnotikz}

% =========  ↓↓↓  À DEFINIR  ↓↓↓  =========
\def\director{Benoît FRESSE}% Le directeur du laboratoire
\def\recipient{professeur Prénom NOM}% Le chercheur à inviter
\def\address{% Son adresse professionnelle
Laboratoire AA\\
Université BB\\
Adresse postale
}
\def\fromdate{JJ mois}% date de début, exemple : 1\up{er} février
\def\todate{JJ mois AAAA}% date de fin, exemple : 15 mars 2030
\def\myself{professeur Mon-prénom MON-NOM}% le nom de l'invitant (vous en général)
\def\research{projet de recherche}% le titre du projet de recherche
% =========  ↑↑↑  À DEFINIR  ↑↑↑  =========

\begin{document}
\pagestyle{empty}
% ---------------------------------------------------------- l'en-t\^ete du labo
\laboheader
% ---------------------------------------------------------- le bas de la page
\labofooter
% ---------------------------------------------------------- le directeur (expéditeur)
\textheader{%
  \textbf{\director}\newline
  Directeur Laboratoire Paul Painlevé -- UMR 8524\newline
  Université de Lille, CNRS\newline
  direction-painleve@univ-lille.fr\newline
  +33 320 434 571
}
% ---------------------------------------------------------- recipient
\vspace*{7mm}
\begin{destinataire}
  \textbf{\recipient}\\
  \address
\end{destinataire}

Villeneuve d'Ascq, le \today
\vspace{1cm}

\renewcommand{\baselinestretch}{1.3}\selectfont
% ============================================================= la lettre
\hspace{4cm}Cher \recipient,
\vspace{4mm}

Je serai ravi de vous accueillir au Laboratoire Paul Painlevé de l’Université de Lille, du \fromdate{} au \todate, afin de mener des recherches avec \myself{} sur le \enquote{\research}. Durant votre séjour, vous pourrez bénéficier des facilités de notre laboratoire (espace de travail, accès internet, accès à la bibliothèque de recherche).
J’espère que vous aurez la possibilité de répondre positivement à cette invitation.
Dans l'attente de recevoir de vos nouvelles, je vous prie d'agréer, cher \recipient, mes sincères salutations.

% ---------------------------------------------------------- la signature du directeur
\vspace{1cm}

\begin{expediteur}[7cm]
  \textbf{/\director/}
\end{expediteur}

\end{document}
